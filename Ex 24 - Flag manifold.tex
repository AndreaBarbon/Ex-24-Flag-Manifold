%
%  Question
%
%  Created by Andrea Barbon on 2011-09-30.
%  Copyright (c) 2011 . All rights reserved.
%
\documentclass[]{article}

% Use utf-8 encoding for foreign characters
\usepackage[utf8]{inputenc}

% Setup for fullpage use
\usepackage{fullpage}

% Uncomment some of the following if you use the features
%
% Running Headers and footers
%\usepackage{fancyhdr}

% Multipart figures
%\usepackage{subfigure}

% Surround parts of graphics with box
\usepackage{boxedminipage}

% Package for including code in the document
\usepackage{listings}

% If you want to generate a toc for each chapter (use with book)
\usepackage{minitoc}

% This is now the recommended way for checking for PDFLaTeX:
\usepackage{ifpdf}

\ifpdf
\usepackage[pdftex]{graphicx}
\else
\usepackage{graphicx}
\fi

% Pacchetti
\usepackage[T1]{fontenc}
\usepackage{pvscript}
\usepackage{amssymb,amsmath}
\usepackage{epigraph}
\usepackage{amsmath}
\usepackage{amsthm}
\usepackage{mathrsfs}
%\usepackage{bbold}

\DeclareMathAlphabet{\mathpzc}{OT1}{pzc}{m}{it}

% Comandi
\newcommand{\C}{\mathbb{C}}
\newcommand{\R}{\mathbb{R}}
\newcommand{\x}{\otimes}
\newcommand{\sab}{\sum_{(a)(b)}}
\newcommand{\e}{\varepsilon}
\newcommand{\D}[2]{\frac{\partial #1}{\partial #2}}
\newcommand{\z}{\bar{z}}
\newcommand{\dd}{\partial}
\newcommand{\la}{\langle}
\newcommand{\ra}{\rangle}
\newcommand{\inn}[2]{\la\; #1 \; ,\; #2 \;\ra}
\newcommand{\gdot}{\dot{\gamma}}

\newcommand{\AND}{\qquad \text{and} \qquad}

\newcommand{\RA}{\rho(A)}
\newcommand{\RO}{\rho_1(A)}
\newcommand{\RT}{\rho_2(A)}
\newcommand{\aplh}{(1-\alpha)}
\newcommand{\Ad}{\ker\text{Ad}}


% Ambienti
\newtheorem{defi}{Definition}[section]
\newtheorem{lem}{Lemma}[section]
\newtheorem{ex}{Example}[section]
\newtheorem{prop}{Proposition}[section]

% Paragrafo
\setlength{\parindent}{0pt}


\title{Exercise 42}
\author{ Andrea Barbon - VU 2206157 }

\date{2012-4-7}

\begin{document}

\ifpdf
\DeclareGraphicsExtensions{.pdf, .jpg, .tif}
\else
\DeclareGraphicsExtensions{.eps, .jpg}
\fi

\maketitle

\section{Part A}
Let $E,F \in \mathpzc{F} $. We can find some linear basis $e_j,f_j$ s.t. 
$$ E_j = \text{span}(e_j) \AND F_j = \text{span}(f_j) $$
and the map $ g $ defined by $e_j \mapsto f_j$ is an element of $G$, and has the property that $ g\cdot E=F $. We therefore see that the action of $\alpha$ is transitive.

\section{Part B}
From what we saw in Part A it is clear that the stabilizer is given by $$ P=\{ \text{diag}(M_j) \mid M_j \in \text{GL}(d_j,\;\mathbb{K}), j=1,\dots,k \}. $$
Using the trivial global chart $\chi:\mathbb{R}^{n^2}\to G$ of $G$ one can see (after a rearranging of the basis of $\mathbb{R}^{n^2}$) that $P$ can be described by $$ P=\chi(\mathbb{R}^{n^2} \cap \{x_1=\dots=x_l=0\}) $$ where the coordinates set to zero are the ones outside the diagonal blocks. It follows by definition that $P$ is a submanifold of $G$, and applying Theorem 9.1 we conclude that $P$ is a closed subset of $G$. Obviously $P$ is a subgroup of $G$, hence a closed subgroup as required.


\section{Part C}
Consider the compact set $\{ (m_1, 0) \} \subset M $, where $m_1 \neq 0$. It's easy to see that the set $$ A \cdot C = \{(e^t m_1,0) \mid t\in A\} = \{(x,0) \mid x\in \mathbb{R}_{>0} \} $$ is not closed. $M$ is locally compact, so we can apply \emph{lemma 11.8} and conclude that the quotient topology $A / M$ is not Hausdorff.


\section{Part D}

In view of \emph{Lemma 11.8}, the condition $(b)$ of \emph{Definition 12.2} is equivalent to the quotient $A / M$ being Hausdorff. In item $(c)$ we showed that this is not the case, hence the action of $A$ on $M$ is not of principal fiber bundle type. 



\end{document}
