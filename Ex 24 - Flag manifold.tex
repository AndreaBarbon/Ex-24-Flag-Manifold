%
%  Question
%
%  Created by Andrea Barbon on 2011-09-30.
%  Copyright (c) 2011 . All rights reserved.
%
\documentclass[]{article}

% Use utf-8 encoding for foreign characters
\usepackage[utf8]{inputenc}

% Setup for fullpage use
\usepackage{fullpage}

% Uncomment some of the following if you use the features
%
% Running Headers and footers
%\usepackage{fancyhdr}

% Multipart figures
%\usepackage{subfigure}

% Surround parts of graphics with box
\usepackage{boxedminipage}

% Package for including code in the document
\usepackage{listings}

% If you want to generate a toc for each chapter (use with book)
\usepackage{minitoc}

% This is now the recommended way for checking for PDFLaTeX:
\usepackage{ifpdf}

\ifpdf
\usepackage[pdftex]{graphicx}
\else
\usepackage{graphicx}
\fi

% Pacchetti
\usepackage[T1]{fontenc}
\usepackage{pvscript}
\usepackage{amssymb,amsmath}
\usepackage{epigraph}
\usepackage{amsmath}
\usepackage{amsthm}
\usepackage{mathrsfs}
%\usepackage{bbold}

\DeclareMathAlphabet{\mathpzc}{OT1}{pzc}{m}{it}

% Comandi
\newcommand{\C}{\mathbb{C}}
\newcommand{\R}{\mathbb{R}}
\newcommand{\x}{\otimes}
\newcommand{\sab}{\sum_{(a)(b)}}
\newcommand{\e}{\varepsilon}
\newcommand{\D}[2]{\frac{\partial #1}{\partial #2}}
\newcommand{\z}{\bar{z}}
\newcommand{\dd}{\partial}
\newcommand{\la}{\langle}
\newcommand{\ra}{\rangle}
\newcommand{\inn}[2]{\la\; #1 \; ,\; #2 \;\ra}
\newcommand{\gdot}{\dot{\gamma}}

\newcommand{\AND}{\qquad \text{and} \qquad}

\newcommand{\RA}{\rho(A)}
\newcommand{\RO}{\rho_1(A)}
\newcommand{\RT}{\rho_2(A)}
\newcommand{\aplh}{(1-\alpha)}
\newcommand{\Ad}{\ker\text{Ad}}
\newcommand{\F}{\mathpzc{F}}



% Ambienti
\newtheorem{defi}{Definition}[section]
\newtheorem{lem}{Lemma}[section]
\newtheorem{ex}{Example}[section]
\newtheorem{prop}{Proposition}[section]

% Paragrafo
\setlength{\parindent}{0pt}


\title{Exercise 42}
\author{ Andrea Barbon - VU 2206157 }

\date{2012-4-7}

\begin{document}

\ifpdf
\DeclareGraphicsExtensions{.pdf, .jpg, .tif}
\else
\DeclareGraphicsExtensions{.eps, .jpg}
\fi

\maketitle

\section{Part A}
Let $E,F \in \mathpzc{F} $. We can find some linear basis $e_j,f_j$ s.t. 
$$ E_j = \text{span}(e_j) \AND F_j = \text{span}(f_j) $$
and the map $ g $ defined by $e_j \mapsto f_j$ is an element of $G$, and has the property that $ g\cdot E=F $. We therefore see that the action of $\alpha$ is transitive.

\section{Part B}
From what we saw in Part A it is clear that the stabilizer is given by $$ P=\{ \text{diag}(M_j) \mid M_j \in \text{GL}(d_j,\;\mathbb{K}), j=1,\dots,k \}. $$
Using the trivial global chart $\chi:\mathbb{R}^{n^2}\to G$ of $G$ one can see (after a rearranging of the basis of $\mathbb{R}^{n^2}$) that $P$ can be described by $$ P=\chi(\mathbb{R}^{n^2} \cap \{x_1=\dots=x_l=0\}) $$ where the coordinates set to zero are the ones outside the diagonal blocks. It follows by definition that $P$ is a submanifold of $G$, and applying Theorem 9.1 we conclude that $P$ is a closed subset of $G$. Obviously $P$ is a subgroup of $G$, hence a closed subgroup as required.


\section{Part C}
We can give $\mathpzc{F}$ a group structure by declaring $ \phi(g) \star \phi(h) = \phi(gh) $. The product $\star$ is well defined on the whole domain since from Part A it follows that $\phi$ is surjective, and it is defined in the right way to make $\phi$ a homomorphism of groups. We can then apply the isomorphism theorem for groups to conclude that such a bijection $\bar{\phi}$ is induced. 


\section{Part D}
To see tht $\phi(K)=\mathpzc{F}$ we look at what we did in Part A, and we observe that using the Grant-Schmidt process we can make all the $e_j$ and the $f_j$ orthonormal, thus making $g$ an orthogonal tranformation. Hence $\forall F \in \mathpzc{F}$ we can find a map $g\in K$ s.t. $\phi(g)=F$, or in other word $\phi(K)=\mathpzc{F}$. \\
From above it follows that the action of $K$ on $\mathpzc{F}$ is transitive, and it is clear that $ H = \text{ker}(\phi_{|K}) = K_E $. Hence we can apply Proposition 15.5 the get the desired diffeomorphism.\\
Finally we know that $K$ is compact and so it is its quotient $K/H$. Therefore $\F$ is compact.

\section{Part E}
We can decompose $G$ as $G=P\cup P^c$, and by looking at $k=e$ we see that $P\subset \text{Im}(m)$. Moreover if $g\in P^c$ then $g\cdot E=F$, and from $\phi(K)=\mathpzc{F}$ we know that $\exists k \in K$ s.t. $ k\cdot E = F$. By the isomorphism fro Part D we deduce that $g$ and $k$ are in the same class, i.e. $\exists p\in P$ s.t. $g=kp$. The surjectivity of $m$ follows.

\section{Part F}
Consider the set $ D=\{d\in\mathpzc{F} \mid d=(k,n-k)\} $, and define the map $$ \psi:D\to G_{n,k}(\mathbb{K}), \qquad d\mapsto F_1. $$ $\psi$ is surjective because $F_1$ can be any linear subspace of dimension $k$, and it is aswell injective because $F_2 = \mathbb{K}^n, F_0 = {0} $ and $F_1=F_1' \implies d=d'$.We covered the general case of $G_{n,k}(\mathbb{K})$, and the projective space is just the special case with $k=1$. 




\end{document}
