%
%  Question
%
%  Created by Andrea Barbon on 2011-09-30.
%  Copyright (c) 2011 . All rights reserved.
%
\documentclass[]{article}

% Use utf-8 encoding for foreign characters
\usepackage[utf8]{inputenc}

% Setup for fullpage use
\usepackage{fullpage}

% Uncomment some of the following if you use the features
%
% Running Headers and footers
%\usepackage{fancyhdr}

% Multipart figures
%\usepackage{subfigure}

% Surround parts of graphics with box
\usepackage{boxedminipage}

% Package for including code in the document
\usepackage{listings}

% If you want to generate a toc for each chapter (use with book)
\usepackage{minitoc}

% This is now the recommended way for checking for PDFLaTeX:
\usepackage{ifpdf}

\ifpdf
\usepackage[pdftex]{graphicx}
\else
\usepackage{graphicx}
\fi

% Pacchetti
\usepackage[T1]{fontenc}
\usepackage{pvscript}
\usepackage{amssymb,amsmath}
\usepackage{epigraph}
\usepackage{amsmath}
\usepackage{amsthm}
\usepackage{mathrsfs}
%\usepackage{bbold}

\DeclareMathAlphabet{\mathpzc}{OT1}{pzc}{m}{it}

% Comandi
\newcommand{\C}{\mathbb{C}}
\newcommand{\R}{\mathbb{R}}
\newcommand{\x}{\otimes}
\newcommand{\sab}{\sum_{(a)(b)}}
\newcommand{\e}{\varepsilon}
\newcommand{\D}[2]{\frac{\partial #1}{\partial #2}}
\newcommand{\z}{\bar{z}}
\newcommand{\dd}{\partial}
\newcommand{\la}{\langle}
\newcommand{\ra}{\rangle}
\newcommand{\inn}[2]{\la\; #1 \; ,\; #2 \;\ra}
\newcommand{\gdot}{\dot{\gamma}}

\newcommand{\AND}{\qquad \text{and} \qquad}

\newcommand{\RA}{\rho(A)}
\newcommand{\RO}{\rho_1(A)}
\newcommand{\RT}{\rho_2(A)}
\newcommand{\aplh}{(1-\alpha)}
\newcommand{\Ad}{\ker\text{Ad}}


% Ambienti
\newtheorem{defi}{Definition}[section]
\newtheorem{lem}{Lemma}[section]
\newtheorem{ex}{Example}[section]
\newtheorem{prop}{Proposition}[section]

% Paragrafo
\setlength{\parindent}{0pt}


\title{Exercise 42}
\author{ Andrea Barbon - VU 2206157 }

\date{2012-4-7}

\begin{document}

\ifpdf
\DeclareGraphicsExtensions{.pdf, .jpg, .tif}
\else
\DeclareGraphicsExtensions{.eps, .jpg}
\fi

\maketitle

\section{Part A}
Let $E,F \in \mathpzc{F} $. We can find some linear basis $e_j,f_j$ s.t. 
$$ E_j = \text{span}(e_j) \AND F_j = \text{span}(f_j) $$
and the map $ g $ defined by $e_j \mapsto \f_j$ is an element of $G$, and has the property that $ g\cdot E=F $. We therefore see that the action of $\alpha$ is transitive.

\section{Part B}
Let $m=(m_1,m_2) \in M$. Since $M$ does not contain the origin, we can assume $m_1 \neq 0$ (if $m_1=0$ than $m_2 \neq 0$, and we can just interchange the names of $m_1$ and $m_2$). We will prove that there exists an $A$-invariant neighborhood $U$ of $m$ such that the restriction of the map $$ \varphi:M\times A \to M\times M, \qquad ((m_1,m_2),t) \mapsto ((m_1,m_2),(e^tm_1,e^{-t}m_2)) $$ to $U\times A$ is proper. We can define $U$ as following
$$ U:= \begin{cases}
	\quad ]0,+\infty[\;\times\;]-\infty,+\infty[ &\text{if} \quad m_1 > 0 \\
	\quad ]-\infty,0[\;\times\;]-\infty,+\infty[ &\text{if} \quad m_1 < 0 \\	
\end{cases} $$
and it is clear that such set is an open neighborhood of $m$ and it is invariant under the action of $A$. We can now define the map $$\psi:U\times U \to U\times A, \qquad ((m_1,m_2), (x,y)) \mapsto ((m_1,m_2), \log(|x|)-\log(|m_1|)) $$ which is well-defined, because $m_1 \neq 0$ and $x \neq 0$. We easily see that this is a local inverse of $\varphi_{| U\times A}$, infact we have 
\begin{eqnarray*}
	\psi\circ\varphi((m_1,m_2),t) &=& \psi((m_1,m_2),(e^tm_1,e^{-t}m_2)) \\\\ 
	&=& ((m_1,m_2),\log(|e^tm_1|)-\log(|m_1|)) \qquad\qquad \text{since}\quad e^t>0 \\\\
	&=& ((m_1,m_2),\log(e^t|m_1|)-\log(|m_1|)) \\\\
	&=& ((m_1,m_2),\log(e^t)+\log(|m_1|)-\log(|m_1|)) \\\\
	&=& ((m_1,m_2),\log(e^t)) \\\\
	&=& ((m_1,m_2),t) \\\\
\end{eqnarray*}
and, restricing to the image $\varphi(U\times A)$,
\begin{eqnarray*}
	\varphi\circ\psi((m_1,m_2),(e^tm_1,e^{-t}m_2)) &=& \varphi((m_1,m_2),\log(|e^tm_1|)-\log(|m_1|)) \\\\ 
	&=& \varphi((m_1,m_2),t) \\\\
	&=& ((m_1,m_2),(e^tm_1,e^{-t}m_2)).
\end{eqnarray*}
Hence we see that $\varphi_{| U\times A}$ has a continuous inverse, which maps compact sets to compact sets. We therefore conclude that $\varphi_{| U\times A}$ is proper. Moreover from \emph{Part A} we know that the action is free, so by \emph{Theorem 13.5} we conclude that the restriction of $A$ to $U$ is of principal fiber bundle type.

\section{Part C}
Consider the compact set $\{ (m_1, 0) \} \subset M $, where $m_1 \neq 0$. It's easy to see that the set $$ A \cdot C = \{(e^t m_1,0) \mid t\in A\} = \{(x,0) \mid x\in \mathbb{R}_{>0} \} $$ is not closed. $M$ is locally compact, so we can apply \emph{lemma 11.8} and conclude that the quotient topology $A / M$ is not Hausdorff.


\section{Part D}

In view of \emph{Lemma 11.8}, the condition $(b)$ of \emph{Definition 12.2} is equivalent to the quotient $A / M$ being Hausdorff. In item $(c)$ we showed that this is not the case, hence the action of $A$ on $M$ is not of principal fiber bundle type. 



\end{document}
